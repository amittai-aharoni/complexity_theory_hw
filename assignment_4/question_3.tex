\begin{question}
    % Rewrite question in correct latex format
    Let  $P_{nu}$ be the class of all decision
    problems $L \subseteq \{0, 1\}^*$
    such that for each $n \in \N$,
    the problem $L \cap \{0, 1\}^n$
    is polynomial-time solvable.
    That is, for each $n \in \N$,
    there exists a deterministic TM $M_n$
    and a constant $c_n$ such that
    $M_n$ decides $L \cap \{0, 1\}^n$ and
    runs in time $O(n^{c_n})$.
    \par In this exercise, you will show that $P_{nu}$
    is different from $P/poly$.
    \par You may opt for an easy variant (A) of this
    exercise, for a total maximum of 11/2 points,
    or a harder variant (B) of this exercise,
    for a total maximum of 2 points.
    \par (A) Prove that if $P_{nu} = P/poly$,
    then the Polynomial Hierarchy collapses.
    \par (B) Prove that $P_{nu} \neq P/poly$.
    \par Hint: for (B), have a look at Sections 6.5 and 6.6 from the book [1].

    \begin{answer} (B).
        It suffices to show that there exists a problem in $P_{nu}$ that cannot 
        be solved by a Turing Machine in $P/poly$. Recall that $P_{nu}$ is a 
        class of all decision problems $L \subseteq {0, 1}^*$ such that for
         each $n \in \mathbb{N}$, the problem $L \cap {0, 1}^n$ is polynomial
          time solvable. By Theorem 6.21 from the book, for every $n > 1$ there 
          exists a function $f:\{0,1\}^n\to\{0,1\}$ that cannot be computed by any 
          circuit $C$ of size 
          $2^n/(10n)$. This also implies that $f$ cannot be computed by any
            polynomial size circuit.
        %    Hence, by this theorem, there is a decision problem 
        %   $L$ for every $n$ in $P_{nu}$ such that $L \cap \set{0, 1}^n$ is 
        %   polynomial time solvable, but there does not exist a \textit{polynomially} 
        %   sized circuit that is able to decide $L$. Thus showing that there 
        %   exists problem in $P_{nu}$ which is not in $P/poly$.
          \par Fix a particular $n > 1$. Let $L$ be the language
          $L=\{x\in\{0,1\}^n \mid f(x)=1\}$. Enumerate all circuits of size
          $n$ by $x_1,\dots,x_{2^n}$
          and define $L'=\{1^k \mid x_k\in L\}$.
          If $L'$ was decideable by a polynomial size circuit $C$ then
          $L$ would be decideable by a polynomial size circuit $C$
          by using our enumeration to encode a string $x$
          to a string $1^k$ and then running $C$.
          Thus $L'$ is not decideable by a polynomial size circuit.
            \par Now we show that $L'$ is in $P_{nu}$ by constructing $M_l$
            for every $l$.
            If $l>2^n$, use the constant $0$ machine.
            Otherwise
            \begin{itemize}
                \item If $x_k\in L$, use the machine that
                verifies that the input is equal to $1^l$.
                \item If $x_k\notin L$, use the constant $0$ machine.
            \end{itemize}
            As all machines are polynomial, $L'\in P_{nu}$.
    \end{answer}
    
\end{question}
