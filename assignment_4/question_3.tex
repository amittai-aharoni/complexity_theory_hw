\begin{question}
    % Rewrite question in correct latex format
    Let  $P_{nu}$ be the class of all decision
    problems $L \subseteq \{0, 1\}^*$
    such that for each $n \in \N$,
    the problem $L \cap \{0, 1\}^n$
    is polynomial-time solvable.
    That is, for each $n \in \N$,
    there exists a deterministic TM $M_n$
    and a constant $c_n$ such that
    $M_n$ decides $L \cap \{0, 1\}^n$ and
    runs in time $O(n^{c_n})$.
    \par In this exercise, you will show that $P_{nu}$
    is different from $P/poly$.
    \par You may opt for an easy variant (A) of this
    exercise, for a total maximum of 11/2 points,
    or a harder variant (B) of this exercise,
    for a total maximum of 2 points.
    \par (A) Prove that if $P_{nu} = P/poly$,
    then the Polynomial Hierarchy collapses.
    \par (B) Prove that $P_{nu} \neq P/poly$.
    \par Hint: for (B), have a look at Sections 6.5 and 6.6 from the book [1].

    \begin{answer} (B).
        It suffices to show that there exists a problem in $P_{nu}$ that cannot be solved by a Turing Machine in $P/poly$. Recall that $P_{nu}$ is a class of all decision problems $L \subseteq \set{0, 1}^*$ such that for each $n \in \mathbb{N}$, the problem $L \cap \set{0, 1}^n$ is polynomial time solvable. By Theorem 6.21 from the book, for every $n > 1$ there exists a function that cannot be computed by a circuit $C$ of size $2^n/(10n)$. Hence, by this theorem, there is a decision problem $L$ for every $n$ in $P_{nu}$ such that $L \cap \set{0, 1}^n$ is polynomial time solvable, but there does not exist a \textit{polynomially} sized circuit that is able to decide $L$. Thus showing that there exists problem in $P_{nu}$ which is not in $P/poly$.
    \end{answer}
\end{question}
